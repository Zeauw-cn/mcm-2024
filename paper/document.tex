\documentclass[11pt,a4paper]{article}

% Basic required LaTeX packages
\usepackage[utf8]{inputenc}
\usepackage[T1]{fontenc}
\usepackage{microtype} % Improves typography
\usepackage{graphicx} % For including graphics
\usepackage{booktabs} % For professional looking tables
\usepackage{float}

% Math packages
\usepackage{amsmath} % Essential for math environments
\usepackage{amssymb} % Additional math symbols
\usepackage{amsthm}  % Theorem environments
\usepackage{mathtools} % Extensions to amsmath

% Define Thms
\newtheorem{theorem}{Theorem}
\newtheorem{definition}[theorem]{Definition}
\newtheorem{corollary}{Corollary}[theorem]
\newtheorem{lemma}[theorem]{Lemma}

% Physics package
\usepackage{physics} % Simplifies many physics notations

% For better handling of hyperlinks
\usepackage[colorlinks=true, allcolors=blue]{hyperref}

% For bibliography
\usepackage{cite} % If you are using BibTeX

% For better captioning
\usepackage{caption}
\usepackage{subcaption}

% For multiple columns
%\usepackage{multicol} % Uncomment if you need multiple columns

% For creating lists
\usepackage{enumitem}

% For improved tables
\usepackage{tabularx}
\title{MCM/C}
\begin{document}
	\maketitle
	\tableofcontents
	
	\section{Background Information}	
	The 2023 Wimbledon Championships marked a significant event in the world of tennis, particularly in the Gentlemen's Singles category. One of the most notable matches was the final between the young Spanish sensation, 20-year-old Carlos Alcaraz, and the seasoned champion, 36-year-old Novak Djokovic. In a surprising turn of events, Alcaraz defeated Djokovic, ending the latter's impressive winning streak at Wimbledon since 2013. This match stood out not just for its outcome but also for its dramatic shifts in momentum, a concept often discussed in sports but hard to quantify. Throughout the match, both players exhibited periods of dominance, with the advantage oscillating between them in an unpredictable manner. This intriguing match, along with other games from the tournament, provides a rich dataset to explore the elusive concept of momentum in tennis and its impact on match outcomes.
	
	Momentum in sports has been studied by many scholars in various fields. Some aim to quantify momentum from a statisitical point of view, while others try to explain it from a sports psychological point of view. All fields are equally valid. However, our team will build a mathematical model to explain and answer the following questions:
	
	\begin{itemize}
		\item Create a model to capture the flow of play in tennis matches, identifying the player performing better at any given time and the extent of their advantage.
		\item Provide a visualization based on your model to depict the match flow.
		\item Use our model to evaluate the claim: "swings in a tennis match are random, as opposed to being influenced by momentum."
		\item Develop a model that predicts when the flow of play is about to change. Identify factors that might be related to these shifts.
		\item Test your model on other matches to evaluate its effectiveness and generalizability.
		\item Produce a report (max 25 pages) with your findings. Include a memo summarizing results and advice for coaches regarding momentum and preparation for matches.
	\end{itemize}
	
	\subsection{Some Info on Tennis Scoring}
	\paragraph{Game}
	\begin{itemize}
		\item Each point won counts as one. 
		\item The first to score 4 points wins the game.
		\item At 3 points each, the score is 'Deuce.' From Deuce, a player must win by two points to win the game. 
		\item In international tennis, the scores of 0, 1, 2, and 3 points are represented by the English words Love, 15, 30, and 40, respectively.
	\end{itemize}
		
	\paragraph{Set}
	\begin{itemize}
		\item The first player to win 6 games wins the set.
		\item If both players win 5 games each, the set is won by the first player to lead by two games.
	\end{itemize}
		
	\paragraph{Tie-break Scoring}
	When the game score in a set reaches 6-all, one of the following tie-break methods is used:
	\begin{itemize}
		\item Long set: One player must lead by two games to win the set.
		\item Tie-break (or 'short set'): Except in the final set (unless otherwise specified), the following rules apply:
		\begin{itemize}
			\item The first player to score 7 points wins the game and the set (at 6-all, a player must win by two points).
			\item The first server serves the first point; thereafter, players alternate serving two consecutive points.
			\item The first point is served from the right court, the second from the left, and the third from the right.
			\item Players change ends after every six points and at the end of the tie-break.
		\end{itemize}
	\end{itemize}
		
	\paragraph{Best of Five Sets Format}
	The match is won by the player who first wins three out of five sets.
	
	\section{Who Has A Bigger Momentum? A Visualisation Approach}
	Before we begins any real development of our model, we need to visualise. Visualisation provides a clear and graphical approach. But, we need to state our definition for "momentum". In one of the academic papers we found, the authors \cite{chen-2020} identified momentum as:
	\begin{definition}[Momentum by Chen et al.]
		Score difference in a given period of time.
	\end{definition}	
	In the paper. Momentum is defined mathematically as:
	$$
	M(t, t_0, \gamma)=\left\{\begin{array}{ll}
		y(t_0+t)-y(t) & \text { if } y(t_0+t)-y(t)>\gamma \\
		0 & \text { otherwise }
	\end{array},\right.
	$$
	where $y(t)$ represents the score difference between the home and visiting team at time $t_0, t$ denotes an increment of game time and $\gamma$ represents the threshold value of momentum. Similarly, the momentum of the other team is
	$$
	M(t, t_0, \gamma)=\left\{\begin{array}{ll}
		y(t_0+t)-y(t) & \text { if } y(t+0+t)-y(t)<-\gamma \\
		0 & \text { otherwise }
	\end{array} .\right.
	$$
	Equipped with this definition, we now proceed to visualisation. 
	\paragraph{Data Preprocessing}
	First we need to process some data in the "Wimbledon\_featured\_matches.csv".
	\begin{itemize}
		\item Convert "match\_id" into unique sequential numbering
		\item Convert all time to seconds
		\item Standarise all scoring into points(e.g. 1,2,3). Reason being nonuniform scoring could affect calculation
	\end{itemize}
	
	Now, based on your definition of momentum, we have generated the following graphs:
	
	%
	%insert graphs here
	%
	
	\begin{figure}[H]
		\centering
		\includegraphics[width=0.7\linewidth]{pics/IMG_2787}
		\caption{placeholder}
		\label{fig:img2787}
	\end{figure}	
	
	We want a model that accounts for more than just score difference. We made 2 considerable additions to the definition of "momentum". 
	\begin{itemize}
		\item Added in "server advantage". Numerous papers have stated that server advantage is an important psychological effect to increase the probability of scoring\cite{Klaassen_1999} \cite{MacPhee_Pollard_2004}, though it alone does not determine the competition outcome. \textbf{We model the effect by adding a "server scoring probability" to whomever is serving.}
		\item Added in "ace". Ace means a server wins a point by serving. This significantly boosts the server's morale. 
	\end{itemize} 
	A modified visualisation shows:
	
	%
	%insert graphs here
	%
	\begin{figure}[H]
		\centering
		\includegraphics[width=0.7\linewidth]{pics/IMG_2787}
		\caption{placeholder}
		\label{fig:img2787}
	\end{figure}
	
	
	\section{Does Swings Happen Randomly?}
	The study of momentum in sports generated great interests among scholars, sports coaches and athletes. The debate is still ongoing--some people do not believen in momentum\cite{Hale_2021}, just like our dear tennis coach.
	
	It was mentioned earlier that the trends of momentum rising and falling reflect performance. We can define the turning point of the rising and falling trends as the inflection point. In each game, by linearly fitting the overall performance of player 1 in the game with the number of balls scored, we can determine the slope that reflects the rising and falling trends. When the performance changes (i.e., when the sign of the slope changes), an inflection point occurs. To filter out the inflection points where the trend changes slowly, only |slope| above a certain threshold are considered as inflection points. These points are marked with red vertical lines on the graph.
	
	In this scenario, we aim to develop a mathematical model to identify turning points in a player's performance trend during a game. The turning points are defined as moments when the trend of performance changes from increasing to decreasing or vice versa, determined by the slope of a fitted line. 
	
	\begin{itemize}
		\item Data Representation
		\begin{itemize}
			\item Let \( t_i \) represent time in the game.
			\item Let \( y_i \) represent the player's score at each \( t_i \).
		\end{itemize}
		\item Linear Regression Model
		\begin{itemize}
			\item We use a simple linear regression model to fit the data points \((t_i, y_i)\).
			\item The linear model is given by \( y = mt + b \), where \( m \) is the slope and \( b \) is the y-intercept. The slope \( m \) indicates the trend of the player's performance. A positive slope implies an increasing trend, while a negative slope indicates a decreasing trend.
			\item To find the best-fit line, we minimize the sum of squared differences between the actual performance scores and the scores predicted by the linear model. The objective is to minimize the Mean Squared Error (MSE), given by:
			\[ MSE = \frac{1}{n} \sum_{i=1}^{n} (y_i - (mt_i + b))^2 \]
		\end{itemize}		
		\item Identifying Turning Points
		\begin{itemize}
			\item A turning point is identified when there is a change in the sign of the slope (from positive to negative or vice versa).
			\item Moreover, to filter out minor fluctuations, a threshold is set for the slope. Only when the absolute value of the slope changes and is greater than a specified threshold (in this case, 0.15) is a turning point acknowledged.
			\item Mathematically, if \( m_{prev} \) and \( m_{current} \) are the slopes of two consecutive segments and \( |m_{current}| > 0.15 \), a turning point is identified when \( \text{sign}(m_{prev}) \neq \text{sign}(m_{current}) \).
		\end{itemize}
	\end{itemize}
	
	Now we have identified all turning points that are actually swings, we need to test for their randomness. We have imposed certain conditions on turning points and the randomness test is as follows:
		
	Let's denote:
	\begin{enumerate}
		\item \( N \) as the total number of games.
		\item \( n_1 \) as the number of games with turning points.
		\item \( n_2 \) as the number of games without turning points.
		\item \( N = n_1 + n_2 \).
		\item \( R \) as the total number of runs (a run is a sequence of consecutive games either all with turning points or all without turning points).
	\end{enumerate}
		
	The expected number of runs (\( E(R) \)) and the variance of the number of runs (\( Var(R) \)) are given by:
	\begin{enumerate}
		\item \[ E(R) = \frac{2n_1n_2}{N} + 1 \]
		\item \[ Var(R) = \frac{2n_1n_2(2n_1n_2 - N)}{N^2(N - 1)} \]
	\end{enumerate}
	
	The test statistic (\( Z \)) is then calculated as:
	\[ Z = \frac{R - E(R)}{\sqrt{Var(R)}} \]
	
	While P-value is given by:
	\[ \text{p-value} = 2P(Z > |z|) \]
	Interpretation	
	\begin{itemize}
		\item If the value of \( Z \) is significantly high or low, it indicates that the turning points are not randomly distributed, suggesting a pattern or trend in the player's performance changes.
		\item A high p-value (usually > 0.05) implies that the turning points are randomly distributed, indicating no specific pattern in performance changes.	
	\end{itemize}
	%
	%insert results as table or graph
	%
	\paragraph{Results}
	We have obtained very low p-values, which suggests that swings are not random.
	
	\bibliography{references}
	\bibliographystyle{plain}
\end{document}